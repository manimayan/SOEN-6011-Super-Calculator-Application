
\documentclass[letterpaper, 11pt]{report}
\usepackage[utf8]{inputenc}
\usepackage{titlesec}
\usepackage{fullpage} % changes the margin
\usepackage{graphicx} %package to manage images
\graphicspath{ {./images/} }

\begin{document}
\begin{titlepage}
\vspace*{0.7in}
\begin{center}
\begin{figure}[htb]
\begin{center}
\includegraphics[width=8cm]{univ_logo}
\end{center}
\end{figure}
\vspace*{0.3in}
\begin{Large}
\textbf{SOEN 6011 : SOFTWARE ENGINEERING PROCESSES} \\
\end{Large}
\vspace*{0.1in}
\begin{Large}
\textbf{SUMMER 2021} \\
\end{Large}
\vspace*{0.9in}
\begin{Large}
\textbf{SUPER CALCULATOR} \\
\end{Large}
\vspace*{0.625in}
\begin{Large} 


\textbf{PROBLEM - 2} \\
Requirements\\\footnotesize{ISO/IEC/IEEE} 29148 Standard \\
\end{Large}
\vspace*{0.625in}
\rule{80mm}{0.1mm}\\
\vspace*{0.1in}
\begin{large}
Authors \\
\vspace*{0.1in}
Rokeya Begum Keya\\
\vspace*{0.1in}
Kyle Taylor Lange\\
\vspace*{0.1in}
Sijie Min\\
\vspace*{0.1in}
Manimaran Palani\\ 
\vspace*{0.3in}
\date{\normalsize\today} 
\end{large}
\end{center}
\begin{center}
https://www.overleaf.com/project/610304de4e6b8d24f7c781b6\end{center}
\end{titlepage}

\newpage
\section*{\centering{PROBLEM 2 - F2: $tan(x)$}}
\normalsize {SOEN 6011 - Summer 2021} \hfill \textbf{Rokeya Begum Keya} \\
\textbf{ Software Engineering Processes}  \hfill \textbf{40183615} \\
\hfill Repository address : https://github.com/Dakatsu/SOEN6011Calculator
\\\\\\
\section*{Assumption:} 
The user will give integer(Degree) value of $x$ in $tan(x)$ function. The value of $tan(x)$ function will be calculated in radian.
\\\
\section*{Requirements:}\cite{ReqView}\cite{29148}\\
The current section describes the requirements to implement the function $tan(x)$.
\\\\\\
\textbf{Requirement Id : F2-R1}\\\\
\begin{tabular}{ll}
\textbf{Overview} & $x = 0^\circ $ in to the $tan(x)$ function \\
\textbf{Version} & 1.0 \\
\textbf{Description} & 
\begin{tabular}[c]{@{}l@{}} If the user gives an input $x = 0^\circ $ for $tan(x)$
\\the function may return 0 as output.
\end{tabular} \\
\textbf{Owner} & Rokeya Begum Keya \\
\textbf{Priority} & High \\
\textbf{Type} & Functional \\
\textbf{Difficulty} & Medium \\
\textbf{Verification Method} & F2\_tanZeroCheck\_1\\                  \end{tabular}
\\\\\\\\\\\\\\\\\\\\\\\\\\\\\\\
\textbf{Requirement Id : F2-R2}\\\\
\begin{tabular}{ll}
\textbf{Overview} & $x = (Positive\ Degree) $ in to the $tan(x)$ function. \\
\textbf{Version} & 1.0 \\
\textbf{Description} & 
\begin{tabular}[c]{@{}l@{}} If the user gives $x = any\ positive\ degree $ for $tan(x)$ 
\\the function may return the approximate value of $tan( positive\ degree)$ 
\\as output.
\end{tabular} \\
\textbf{Owner} & Rokeya Begum Keya \\
\textbf{Priority} & High \\
\textbf{Type} & Functional \\
\textbf{Difficulty} & Medium \\
\textbf{Verification Method} & F2\_tanFortyCheck\_2\\           \end{tabular}
\\\\\\\
\textbf{Requirement Id : F2-R3}\\\\
\begin{tabular}{ll}
\textbf{Overview} & $x = 90^\circ $ in to the $tan(x)$ function \\
\textbf{Version} & 1.0 \\
\textbf{Description} & 
\begin{tabular}[c]{@{}l@{}} If the user gives an input $x = 90^\circ $ for $tan(x)$
\\the function may return "undefined" as output.
\end{tabular} \\
\textbf{Owner} & Rokeya Begum Keya \\
\textbf{Priority} & High \\
\textbf{Type} & Functional \\
\textbf{Difficulty} & Medium \\
\textbf{Verification Method} & F2\_tanNinetyCheck\_3\\                 \end{tabular}
\\\\\\\\\\
\textbf{Requirement Id : F2-R4}\\\\
\begin{tabular}{ll}
\textbf{Overview} & $x = (Negative\ or\ Positive\ Degree) $ in to the $tan(x)$ function\\
\textbf{Description} & 
\begin{tabular}[c]{@{}l@{}} If the user gives $x = any\ Negative\ or\ Positive\ degree $ 
\\for which  $tan(x)$ value is Negative
\\the function may return the approximate negative value of 
\\$tan(Negative\ or\ Positive\ Degree)$ 
\\as output.
\end{tabular} \\
\textbf{Owner} & Rokeya Begum Keya \\
\textbf{Priority} & High \\
\textbf{Type} & Functional \\
\textbf{Difficulty} & Medium \\
\textbf{Verification Method} &F2\_tanNegativeValueCheck\_4\\                  \end{tabular}
\\\\\\\\\\\\\\\\\\\
\textbf{Requirement Id : F2-R5}\\\\
\begin{tabular}{ll}
\textbf{Overview} & $x = (Negative\ Degree) $ in to the $tan(x)$ function\\
\textbf{Version} & 1.0 \\
\textbf{Description} & 
\begin{tabular}[c]{@{}l@{}} If the user gives $x = any\ Negative\ degree $ for $tan(x)$ 
\\the function may return the approximate value of $tan( Negative\ degree)$ 
\\as output.
\end{tabular} \\
\textbf{Owner} & Rokeya Begum Keya \\
\textbf{Priority} & High \\
\textbf{Type} & Functional \\
\textbf{Difficulty} & Medium \\
\textbf{Verification Method} & F2\_tanNegativeNumberCheck\_5\\                                       \end{tabular}
\\\\\\\\\
\textbf{Requirement Id : F2-R6}\\\\
\begin{tabular}{ll}
\textbf{Overview} & $x = 180^\circ $ in to the $tan(x)$ function \\
\textbf{Version} & 1.0 \\
\textbf{Description} & 
\begin{tabular}[c]{@{}l@{}} If the user gives an input $x = 180^\circ $ for $tan(x)$
\\the function may return 0 as output.
\end{tabular} \\
\textbf{Owner} & Rokeya Begum Keya \\
\textbf{Priority} & High \\
\textbf{Type} & Functional \\
\textbf{Difficulty} & Medium \\
\textbf{Verification Method} & F2\_tanOneHundredAndEightyCheck\_6\\     \end{tabular}
\\\\\\\\\\
\textbf{Requirement Id : F2-R7}\\\\
\begin{tabular}{ll}
\textbf{Overview} & $x = 90^\circ $ in to the $Rad(x)$  \\
\textbf{Version} & 1.0 \\
\textbf{Description} & 
\begin{tabular}[c]{@{}l@{}} If the user gives an input $x = 90^\circ $ for $Rad(x)$
\\the function may return the approximate value in radian as output.
\end{tabular} \\
\textbf{Owner} & Rokeya Begum Keya \\
\textbf{Priority} & High \\
\textbf{Type} & Functional \\
\textbf{Difficulty} & Medium \\
\textbf{Verification Method} & F2\_getRadCheck\_7\\    
          \end{tabular}
\\\\\\\\\\\\\\\\\\\
\textbf{Requirement Id : F2-R8}\\\\
\begin{tabular}{ll}
\textbf{Overview} & $x = 180^\circ $ in to the $Rad(x)$  \\
\textbf{Version} & 1.0 \\
\textbf{Description} & 
\begin{tabular}[c]{@{}l@{}} If the user gives an input $x = 180^\circ $ for $Rad(x)$
\\the function may return the approximate value $(3.14159..)$ in radian  as output.
\end{tabular} \\
\textbf{Owner} & Rokeya Begum Keya \\
\textbf{Priority} & High \\
\textbf{Type} & Functional \\
\textbf{Difficulty} & Medium \\
\textbf{Verification Method} & F2\_getRadOneHundredAndEightyCheck\_8\\          \end{tabular}
\\\\\\\\\\\\
\textbf{Requirement Id : F2-R9}\\\\
\begin{tabular}{ll}
\textbf{Overview} & $x = 0^\circ $ in to the $sin(x)$ function \\
\textbf{Version} & 1.0 \\
\textbf{Description} & 
\begin{tabular}[c]{@{}l@{}} If the user gives an input $x = 0^\circ $ for $sin(x)$
\\the function may return 0 as output.
\end{tabular} \\
\textbf{Owner} & Rokeya Begum Keya \\
\textbf{Priority} & High \\
\textbf{Type} & Functional \\
\textbf{Difficulty} & Medium \\
\textbf{Verification Method} & F2\_getSinZeroCheck\_9\\                  \end{tabular}
\\\\\\\\\\
\textbf{Requirement Id : F2-R10}\\\\
\begin{tabular}{ll}
\textbf{Overview} & $x = (Positive\ Degree) $ in to the $sin(x)$ function. \\
\textbf{Version} & 1.0 \\
\textbf{Description} & 
\begin{tabular}[c]{@{}l@{}} If the user gives $x = any\ positive\ degree $ for $sin(x)$ 
\\the function may return the approximate value of $sin( positive\ degree)$
\\as output.
\end{tabular} \\
\textbf{Owner} & Rokeya Begum Keya \\
\textbf{Priority} & High \\
\textbf{Type} & Functional \\
\textbf{Difficulty} & Medium \\
\textbf{Verification Method} & F2\_getSinFortyCheck\_10\\                   \end{tabular}
\\\\\\\\\\
\textbf{Requirement Id : F2-R11}\\\\
\begin{tabular}{ll}
\textbf{Overview} & $x = 0^\circ $ in to the $cos(x)$ function \\
\textbf{Version} & 1.0 \\
\textbf{Description} & 
\begin{tabular}[c]{@{}l@{}} If the user gives an input $x = 0^\circ $ for $cos(x)$
\\the function may return 1 as output.
\end{tabular} \\
\textbf{Owner} & Rokeya Begum Keya \\
\textbf{Priority} & High \\
\textbf{Type} & Functional \\
\textbf{Difficulty} & Medium \\
\textbf{Verification Method} & F2\_getCosZeroCheck\_11\\                  \end{tabular}
\\\\\\\\\\\
\textbf{Requirement Id : F2-R12}\\\\
\begin{tabular}{ll}
\textbf{Overview} & $x = (Positive\ Degree) $ in to the $cos(x)$ function. \\
\textbf{Version} & 1.0 \\
\textbf{Description} & 
\begin{tabular}[c]{@{}l@{}} If the user gives $x = any\ positive\ degree $ for $cos(x)$ 
\\the function may return the approximate value of $cos( positive\ degree)$
\\as output.
\end{tabular} \\
\textbf{Owner} & Rokeya Begum Keya \\
\textbf{Priority} & High \\
\textbf{Type} & Functional \\
\textbf{Difficulty} & Medium \\
\textbf{Verification Method} & F2\_getCosFortyCheck\_12\\                   \end{tabular}
\\\\\\\\\\
\textbf{Requirement Id : F2-R13}\\\\
\begin{tabular}{ll}
\textbf{Overview} & Availability \\
\textbf{Version} & 1.0 \\
\textbf{Description} & 
\begin{tabular}[c]{@{}l@{}} The system may provide the calculation to the user within four seconds.
\end{tabular} \\
\textbf{Owner} & Rokeya Begum Keya \\
\textbf{Priority} & High \\
\textbf{Type} & Non-Functional \\
\textbf{Difficulty} & Medium \\
                \end{tabular}
\pagebreak

\section*{\centering{PROBLEM 2 - F3: Hyperbolic Sine, $sinh(x)$}}
\normalsize {SOEN 6011 - Summer 2021} \hfill \textbf{Kyle Taylor Lange} \\
\textbf{ Software Engineering Processes}  \hfill \textbf{27627696} \\
\hfill Repository address : https://github.com/Dakatsu/SOEN6011Calculator
\\
\section{Function Requirements}
\section*{Requirements and Assumptions}
The current section describes the requirements and assumptions to implement the function $sinh(x)$.
\\\\
\textbf{Explicit Assumptions:}  For a value to be \textit{accurate}, it shall be correct up to a specific number of decimal places. This value will be based on balancing accuracy with computation speed. Above a certain absolute input value, the function may return $\pm infinity$ due to the exponential nature of the function going beyond the range of a double floating point.
\\\\

\textbf{Requirement ID: F3-R1}\\\\
\begin{tabular}{ll}
\textbf{Overview}            & Input $x$ into $sinh(x)$ function.
\\\textbf{Version}             & 1
\\\textbf{Description}         & \begin{tabular}[c]{@{}l@{}}If the $sinh(x)$ function as given in Problem 1 receives an\\ integer or decimal input $x$, the system shall return an accurate value.\end{tabular}
\\\textbf{Owner}               & Kyle Taylor Lange
\\\textbf{Priority}            & High
\\\textbf{Type}                & Functional
\\\textbf{Difficulty}          & High
\\\textbf{Verification Method} &                                                                     
\end{tabular}
\\\\

\textbf{Requirement ID: F3-R2}\\\\
\begin{tabular}{ll}
\textbf{Overview}            & Availability
\\\textbf{Version}             & 1
\\\textbf{Description}         & \begin{tabular}[c]{@{}l@{}}The system may provide the calculation in F3-R1\\ to the user within three seconds.\end{tabular}
\\\textbf{Owner}               & Kyle Taylor Lange
\\\textbf{Priority}            & High
\\\textbf{Type}                & Performance
\\\textbf{Difficulty}          & Medium
\\\textbf{Verification Method} &                                                                     
\end{tabular}

\pagebreak

\section*{\centering{PROBLEM 2 - F5}}
\normalsize {SOEN 6011 - Summer 2021} \hfill \textbf{Sijie Min} \\
\textbf{ Software Engineering Processes}  \hfill \textbf{40152234} \\
\hfill Repository address : https://github.com/Dakatsu/SOEN6011Calculator
\\\\\\\\\\
\section*{Requirements and Assumptions}
The user will give value ofa,b and x   .a,b can be both integer and decimal, x is integer

\textbf{Requirement Id : F5-R1}\\\\
\begin{tabular}{ll}
\textbf{Overview} & 	Sets a,b, then input x into  \(y=ab^x\) \\
\textbf{Version} & 1.0 \\
\textbf{Description} & 
\begin{tabular}[c]{@{}l@{}} If a is set to 0, the output result is 0. If the x input is 0, the return result is equal to a.
\end{tabular} \\
\textbf{Owner} & Sijie Min \\
\textbf{Priority} & High \\
\textbf{Type} & Functional \\
\textbf{Difficulty} & Medium \\
\textbf{Verification Method} &F5test \\
   \end{tabular}

\textbf{Requirement Id : F5-R2}\\\\
\begin{tabular}{ll}
\textbf{Overview} & 	Sets a,b, then input x into  \(y=ab^x\) \\
\textbf{Version} & 1.0 \\
\textbf{Description} & 
\begin{tabular}[c]{@{}l@{}} x can be entered as a positive number or a negative number.
\end{tabular} \\
\textbf{Owner} & Sijie Min \\
\textbf{Priority} & High \\
\textbf{Type} & Functional \\
\textbf{Difficulty} & Medium \\
\textbf{Verification Method} &F5test \\
   \end{tabular}

\textbf{Requirement Id : F5-R3}\\\\
\begin{tabular}{ll}
\textbf{Overview} & 	Availability \\
\textbf{Version} & 1.0 \\
\textbf{Description} & 
\begin{tabular}[c]{@{}l@{}} Avoid calculating x times of multiplications, but reduce the number of multiplications \\to approximately \(\sqrt x\)  times to increase the running speed.
\end{tabular} \\
\textbf{Owner} & Sijie Min \\
\textbf{Priority} & High \\
\textbf{Type} & Functional \\
\textbf{Difficulty} & High \\
\textbf{Verification Method} &F5test \\
   \end{tabular}

\textbf{Requirement Id : F5-R4}\\\\
\begin{tabular}{ll}
\textbf{Overview} & 	Availability \\
\textbf{Version} & 2.0 \\
\textbf{Description} & 
\begin{tabular}[c]{@{}l@{}} Ensure the accuracy of decimal operations to the power of decimals 
\end{tabular} \\
\textbf{Owner} & Sijie Min \\
\textbf{Priority} & High \\
\textbf{Type} & Functional \\
\textbf{Difficulty} & High \\
\textbf{Verification Method} &F5test \\
   \end{tabular}
 \begin{thebibliography}{}
 
\bibitem{test1}
Mike Spivey. "The fuzz Manual" Manual and software copyright. J. M. Spivey 1988, 1992, 2000


\end{thebibliography}   
 
\pagebreak

\section*{\centering{PROBLEM 2 - F7 : \(x^y\)}}
\normalsize {SOEN 6011 - Summer 2021} \hfill \textbf{Manimaran Palani} \\
\textbf{ Software Engineering Processes}  \hfill \textbf{40167543} \\
\hfill Repository address : https://github.com/Dakatsu/SOEN6011Calculator
\\
\section*{Requirements and Assumptions}\cite{ReqView}\cite{29148}\\
The current section describes the requirements and assumptions to implement the function \(x^y\).
\\\\\\
\textbf{Explicit Assumption :}  The transcendental function \(x^y\) will be accurate and accepts input which comprises of rational and irrational numbers.
\\\\\\\\
\textbf{Requirement Id : F7-R1}\\\\
\begin{tabular}{ll}
\textbf{Overview} & X(0) to the power of Y(0) \\
\textbf{Version} & 1.0 \\
\textbf{Description} & 
\begin{tabular}[c]{@{}l@{}}If the user gives an input for X as Zero and input for Y as Zero.\\The function may return the 1 as output.
\end{tabular} \\
\textbf{Owner} & Manimaran Palani \\
\textbf{Priority} & High \\
\textbf{Type} & Functional \\
\textbf{Difficulty} & Medium \\
\textbf{Verification Method} & F7\_TestCase\_1\\
\end{tabular}
\\\\\\\\\\
\textbf{Requirement Id : F7-R2}\\\\
\begin{tabular}{ll}
\textbf{Overview} & X(0) to the power of Y(Real Number) \\
\textbf{Version} & 1.0 \\
\textbf{Description} & 
\begin{tabular}[c]{@{}l@{}}If the user gives an input for X as zero and input for Y as
\\any Real Number. The function may return zero as output.
\end{tabular} \\
\textbf{Owner} & Manimaran Palani \\
\textbf{Priority} & High \\
\textbf{Type} & Functional \\
\textbf{Difficulty} & Medium \\
\textbf{Verification Method} & F7\_TestCase\_2\\                                   \end{tabular}
\\\\\\\\\\\\\\\\
\textbf{Requirement Id : F7-R3}\\\\
\begin{tabular}{ll}
\textbf{Overview} & X(Positive Number) to the power of Y(0) \\
\textbf{Version} & 1.0 \\
\textbf{Description} & 
\begin{tabular}[c]{@{}l@{}}If the user gives an input for X of any positive number and\\input for Y as Zero.The function may return 1 as the output.
\end{tabular} \\
\textbf{Owner} & Manimaran Palani \\
\textbf{Priority} & High \\
\textbf{Type} & Functional \\
\textbf{Difficulty} & Medium \\
\textbf{Verification Method} & F7\_TestCase\_3\\                                                         \end{tabular}
\\\\\\\\\\\\\\\\
\textbf{Requirement Id : F7-R4}\\\\
\begin{tabular}{ll}
\textbf{Overview} & X(Negative Number) to the power of Y (0) \\
\textbf{Version} & 1.0 \\
\textbf{Description} & 
\begin{tabular}[c]{@{}l@{}}If the user gives an input for X of any negative number and
\\input for Y as Zero.The function may return 1 as the output.
\end{tabular} \\
\textbf{Owner} & Manimaran Palani \\
\textbf{Priority} & High \\
\textbf{Type} & Functional \\
\textbf{Difficulty} & Medium \\
\textbf{Verification Method} & F7\_TestCase\_4\\
\end{tabular}
\\\\\\\\\\\\\\\\
\textbf{Requirement Id : F7-R5}\\\\
\begin{tabular}{ll}
\textbf{Overview} & X(Positive Number) to the power of Y(1) \\
\textbf{Version} & 1.0 \\
\textbf{Description} & 
\begin{tabular}[c]{@{}l@{}}If the user gives an input for X as any positive number and input \\for Y as 1. The function may return X as the output.
\end{tabular} \\
\textbf{Owner} & Manimaran Palani \\
\textbf{Priority} & High \\
\textbf{Type} & Functional \\
\textbf{Difficulty} & Medium \\
\textbf{Verification Method} & F7\_TestCase\_5\\
\end{tabular}
\\\\\\\\\\\\\\
\textbf{Requirement Id : F7-R6}\\\\
\begin{tabular}{ll}
\textbf{Overview} & X(Positive Number) to the power of Y(Positive Number) \\
\textbf{Version} & 1.0 \\
\textbf{Description} & \begin{tabular}[c]{@{}l@{}}If the user gives an input for X as any positive number and input \\for Y as positive number. The function may return positive \\number as the output.\end{tabular} \\
\textbf{Owner} & Manimaran Palani \\
\textbf{Priority} & High \\
\textbf{Type} & Functional \\
\textbf{Difficulty} & Medium \\
\textbf{Verification Method} & F7\_TestCase\_6\\                                    \end{tabular}
\\\\\\\\\\\\
\textbf{Requirement Id : F7-R7}\\\\
\begin{tabular}{ll}
\textbf{Overview} & X(Negative Number) to the power of Y(Positive Even Number) \\
\textbf{Version} & 1.0 \\
\textbf{Description} & \begin{tabular}[c]{@{}l@{}}If the user gives an input for X as any Negative number and input \\for Y as positive Even number. The function may return positive \\number as the output.\end{tabular} \\
\textbf{Owner} & Manimaran Palani \\
\textbf{Priority} & High \\
\textbf{Type} & Functional \\
\textbf{Difficulty} & Medium \\
\textbf{Verification Method} & F7\_TestCase\_6\\                                    \end{tabular}
\\\\\\\\\\\\
\textbf{Requirement Id : F7-R8}\\\\
\begin{tabular}{ll}
\textbf{Overview} & X(Negative Number) to the power of Y(Positive Odd Number) \\
\textbf{Version} & 1.0 \\
\textbf{Description} & \begin{tabular}[c]{@{}l@{}}If the user gives an input for X as any negative number and input \\for Y as positive odd number. The function may return negative \\number as the output.\end{tabular} \\
\textbf{Owner} & Manimaran Palani \\
\textbf{Priority} & High \\
\textbf{Type} & Functional \\
\textbf{Difficulty} & Medium \\
\textbf{Verification Method} & F7\_TestCase\_6\\                                    \end{tabular}
\\\\\\\\\\\\
\textbf{Requirement Id : F7-R9}\\\\
\begin{tabular}{ll}
\textbf{Overview} & Availability \\
\textbf{Version} & 1.0 \\
\textbf{Description} & 
\begin{tabular}[c]{@{}l@{}}The system may provide the response with output to the user \\ within finite time. \\
\end{tabular} \\
\textbf{Owner} & Manimaran Palani \\
\textbf{Priority} & High \\
\textbf{Type} & Non-Functional \\
\textbf{Difficulty} & Medium \\
\end{tabular}
\begin{thebibliography}{}
\bibitem{ReqView} 
ReqView : Nykamp DQ: Requirements Specification Templates
\\\texttt{https://www.reqview.com/doc/iso-iec-ieee-29148-templates}
\bibitem{29148} 
29148-2018-ISO/IEC/IEEE International Standard-Systems and software engineering-Life cycle processes-Requirements engineering,
\\\texttt{https://standards.ieee.org/standard/29148-2018.html}
\end{thebibliography}
\end{document}