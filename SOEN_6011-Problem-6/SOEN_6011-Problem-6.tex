\documentclass[letterpaper, 11pt]{report}
\usepackage[utf8]{inputenc}
\usepackage{titlesec}
\usepackage{fullpage} % changes the margin
\usepackage{graphicx} %package to manage images
\graphicspath{ {./images/} }

\begin{document}
\begin{titlepage}
\vspace*{0.7in}
\begin{center}
\begin{figure}[htb]
\begin{center}
\includegraphics[width=8cm]{univ_logo}
\end{center}
\end{figure}
\vspace*{0.3in}
\begin{Large}
\textbf{SOEN 6011 : SOFTWARE ENGINEERING PROCESSES} \\
\end{Large}
\vspace*{0.1in}
\begin{Large}
\textbf{SUMMER 2021} \\
\end{Large}
\vspace*{0.9in}
\begin{Large}
\textbf{SUPER CALCULATOR} \\
\end{Large}
\vspace*{0.9in}
\begin{Large} 


\textbf{PROBLEM - 6} \\
Unit Test Cases\\
\end{Large}
\vspace*{0.625in}
\rule{80mm}{0.1mm}\\
\vspace*{0.1in}
\begin{large}
Authors \\
\vspace*{0.1in}
Rokeya Begum Keya\\
\vspace*{0.1in}
Kyle Taylor Lange\\
\vspace*{0.1in}
Sijie Min\\
\vspace*{0.1in}
Manimaran Palani\\ 
\vspace*{0.3in}
\date{\normalsize\today} 
\end{large}
\end{center}
\begin{center}
https://www.overleaf.com/project/610304de4e6b8d24f7c781b6\end{center}
\end{titlepage}
\tableofcontents
\newpage
\addcontentsline{toc}{section}{a) Description on Unit Test Cases }
\newpage
\pagebreak
\section*{Unit Test Cases Description}
\section*{\centering{PROBLEM 6 - F2: $tan(x)$}}
\normalsize {SOEN 6011 - Summer 2021} \hfill \textbf{Rokeya Begum Keya} \\
\textbf{ Software Engineering Processes}  \hfill \textbf{40183615} \\
\hfill Repository address : https://github.com/Dakatsu/SOEN6011Calculator
\\\\\\
\section*{Unit Test Case for F2 Function}
The unit test cases for $tan(x)$ function is done using \textbf{JUnit 4} which are traceable to the requirements in problem-2.\\\\\\
\textbf{Test Case : F2\_UnitTestCase\_1}\\\\
\begin{tabular}{ll}
\textbf{Test Case ID} & F2\_tanZeroCheck\_1\\
\textbf{Requirement ID} & F2-R1 \\
\textbf{Action} & 
\begin{tabular}[c]{@{}l@{}}The user clicks the button "Tan" and gives an input 0 (degree) and
\\then click result(=) button. \\
\end{tabular} \\
\textbf{Input(s) } & $tan(0)$ \\
\textbf{Expected Output } & 0 \\
\textbf{Actual Output } &   0 \\
\textbf{Test Result } & Success \\
\end{tabular}
\\\\\\\\
\textbf{Test Case : F2\_UnitTestCase\_2}\\\\
\begin{tabular}{ll}
\textbf{Test Case ID} & F2\_tanFortyCheck\_2\\
\textbf{Requirement ID} & F2-R2 \\
\textbf{Action} & 
\begin{tabular}[c]{@{}l@{}}The user clicks the button "Tan" and gives an input 40 (degree) and
\\then click result(=) button. \\
\end{tabular} \\
\textbf{Input(s) } & $tan(40)$ \\
\textbf{Expected Output } & 0.83910101 \\
\textbf{Actual Output } & 0.83910101 \\
\textbf{Test Result } & Success \\
\end{tabular}
\\\\\\\\\\\\\\\\\\\\\
\textbf{Test Case : F2\_UnitTestCase\_3}\\\\
\begin{tabular}{ll}
\textbf{Test Case ID} & F2\_tanNinetyCheck\_3\\
\textbf{Requirement ID} & F2-R3 \\
\textbf{Action} & 
\begin{tabular}[c]{@{}l@{}}The user clicks the button "Tan" and gives an input 90 (degree) and
\\then click result(=) button. \\
\end{tabular} \\
\textbf{Input(s) } & $tan(90)$ \\
\textbf{Expected Output } & undefined \\
\textbf{Actual Output } &
undefined \\
\textbf{Test Result } & Success \\
\end{tabular}
\\\\\\\\
\textbf{Test Case : F2\_UnitTestCase\_4}\\\\
\begin{tabular}{ll}
\textbf{Test Case ID} & F2\_tanNegativeValueCheck\_4\\
\textbf{Requirement ID} & F2-R4 \\
\textbf{Action} & 
\begin{tabular}[c]{@{}l@{}}The user clicks the button "Tan" and gives an input 95 (degree) and
\\then click result(=) button. \\
\end{tabular} \\
\textbf{Input(s) } & $tan(95)$ \\
\textbf{Expected Output } & -11.43005230 \\
\textbf{Actual Output } &
-11.43005230 \\
\textbf{Test Result } & Success \\
\end{tabular}
\\\\\\\\
\textbf{Test Case : F2\_UnitTestCase\_5}\\\\
\begin{tabular}{ll}
\textbf{Test Case ID} & F2\_tanNegativeNumberCheck\_5 \\
\textbf{Requirement ID} & F2-R5 \\
\textbf{Action} & 
\begin{tabular}[c]{@{}l@{}}The user clicks the button "Tan" and gives an input -10 (degree) and
\\then click result(=) button. \\
\end{tabular} \\
\textbf{Input(s) } & $tan(-10)$ \\
\textbf{Expected Output } & -0.17723233 \\
\textbf{Actual Output } &
-0.17723233 \\
\textbf{Test Result } & Success \\
\end{tabular}
\\\\\\\\\\\\\\\\\\\\\\\
\textbf{Test Case : F2\_UnitTestCase\_6}\\\\
\begin{tabular}{ll}
\textbf{Test Case ID} & F2\_tanOneHundredAndEightyCheck\_6 \\
\textbf{Requirement ID} & F2-R6 \\
\textbf{Action} & 
\begin{tabular}[c]{@{}l@{}}The user clicks the button "Tan" and gives an input 180 (degree) and
\\then click result(=) button. \\
\end{tabular} \\
\textbf{Input(s) } & $tan(180)$ \\
\textbf{Expected Output } & 0 \\
\textbf{Actual Output } &   0 \\
\textbf{Test Result } & Success \\
\end{tabular}
\\\\\\\\
\textbf{Test Case : F2\_UnitTestCase\_7}\\\\
\begin{tabular}{ll}
\textbf{Test Case ID} & F2\_getRadCheck\_7 \\
\textbf{Requirement ID} & F2-R7 \\
\textbf{Action} & 
\begin{tabular}[c]{@{}l@{}}To make sure that radian function in $tan(x)$ is working properly,
\\I had to do the unit test of Rad(x) and gives an input for $x$ = 90 (degree).
\end{tabular} \\
\textbf{Input(s) } & $Rad(90)$ \\
\textbf{Expected Output } & 1.57079633 \\
\textbf{Actual Output } & 1.57079633 \\
\textbf{Test Result } & Success \\
\end{tabular}
\\\\\\\\
\textbf{Test Case : F2\_UnitTestCase\_8}\\\\
\begin{tabular}{ll}
\textbf{Test Case ID} & F2\_getRadOneHundredAndEightyCheck\_8\\
\textbf{Requirement ID} & F2-R8 \\
\textbf{Action} & 
\begin{tabular}[c]{@{}l@{}}To make sure that radian function in $tan(x)$ is working properly,
\\I had to do the unit test of Rad(x) and gives an input for $x$ = 180 (degree).
\end{tabular} \\
\textbf{Input(s) } & $Rad(180)$ \\
\textbf{Expected Output } & 3.14159 \\
\textbf{Actual Output } & 3.14159 \\
\textbf{Test Result } & Success \\
\end{tabular}
\\\\\\\\\\\\\\\\\\\\\\\\\\\
\textbf{Test Case : F2\_UnitTestCase\_9}\\\\
\begin{tabular}{ll}
\textbf{Test Case ID} & F2\_getSinZeroCheck\_9 \\
\textbf{Requirement ID} & F2-R9 \\
\textbf{Action} & 
\begin{tabular}[c]{@{}l@{}}To make sure that $sin(x)$ function for $tan(x)$ is working properly,
\\I had to do the unit test of $sin(x)$ function and gives an input for 0 (degree).
\end{tabular} \\
\textbf{Input(s) } & $sin(0)$ \\
\textbf{Expected Output } & 0.0 \\
\textbf{Actual Output } &   0.0 \\
\textbf{Test Result } & Success \\
\end{tabular}
\\\\\\\\
\textbf{Test Case : F2\_UnitTestCase\_10}\\\\
\begin{tabular}{ll}
\textbf{Test Case ID} & F2\_getSinFortyCheck\_10\\
\textbf{Requirement ID} & F2-R10 \\
\textbf{Action} & 
\begin{tabular}[c]{@{}l@{}}To make sure that $sin(x)$ function for $tan(x)$ is working properly,
\\I had to do the unit test of $sin(x)$ function and gives an input for 40 (degree).
\end{tabular} \\
\textbf{Input(s) } & $sin(40)$ \\
\textbf{Expected Output } & 0.642788 \\
\textbf{Actual Output } & 0.642788 \\
\textbf{Test Result } & Success \\
\end{tabular}
\\\\\\\\
\textbf{Test Case : F2\_UnitTestCase\_11}\\\\
\begin{tabular}{ll}
\textbf{Test Case ID} & F2\_getCosZeroCheck\_11\\
\textbf{Requirement ID} & F2-R11 \\
\textbf{Action} & 
\begin{tabular}[c]{@{}l@{}}To make sure that $cos(x)$ function for $tan(x)$ is working properly,
\\I had to do the unit test of $cos(x)$ function and gives an input for 0 (degree).
\end{tabular} \\
\textbf{Input(s) } & $cos(0)$ \\
\textbf{Expected Output } & 1 \\
\textbf{Actual Output } &   1 \\
\textbf{Test Result } & Success \\
         \end{tabular}
\\\\\\\\\\\\\\\\\\\\\\\\\\\\\
\textbf{Test Case : F2\_UnitTestCase\_12}\\\\
\begin{tabular}{ll}
\textbf{Test Case ID} & F2\_getCosFortyCheck\_12\\
\textbf{Requirement ID} & F2-R12 \\
\textbf{Action} & 
\begin{tabular}[c]{@{}l@{}}To make sure that $cos(x)$ function for $tan(x)$ is working properly,
\\I had to do the unit test of $cos(x)$ function and gives an input for 40 (degree).
\end{tabular} \\
\textbf{Input(s) } & $cos(40)$ \\
\textbf{Expected Output } & 0.76604305 \\
\textbf{Actual Output } & 0.76604305 \\
\textbf{Test Result } & Success \\
\end{tabular}
\\\\\\\\\
\begin{center}
\includegraphics[width= 10cm]{UnitTestCaseF2}
  \begin{center}
Figure: Unit Testing results for tangent function $ (tan(x))$\end{center}
\end{center}
\\\\\\\\

\pagebreak

\section*{\centering{PROBLEM 6 - F3: Hyperbolic Sine, $sinh(x)$}}
\normalsize {SOEN 6011 - Summer 2021} \hfill \textbf{Kyle Taylor Lange} \\
\textbf{ Software Engineering Processes}  \hfill \textbf{27627696} \\
\hfill Repository address : https://github.com/Dakatsu/SOEN6011Calculator
\\\\\\
A variety of JUnit 5 tests were created in $SinhLibrariesTest.java$ to test the quality of the $sinh$ function. These were made as atomically as possible per guidelines on writing unit tests. For example, one unit test ensures that $sinh(0)$ returns 0, while another ensures that $sinh(1)$ returns ~1.175. Despite it not being explicitly required, the subordinate functions also have unit tests. This is valuable since the $sinh$ function largely depends on them for its level of accuracy, and it may allow an incorrect result for $sinh$ to be immediately traced to a change in a subordinate function. \\\\

There were two requirements, which quickly summarized are that the function returns accurate values according to the equation in problem 1, and that the function may return the result within three seconds. Only this former requirement has unit tests since it is inadvisable to make unit tests to ensure something happens within a specific period of time. Tests that do so could randomly fail or differ between machines, which goes against the purpose and guidelines for writing unit tests.

\pagebreak

\section*{\centering{PROBLEM 6 - F5}}
\normalsize {SOEN 6011 - Summer 2021} \hfill \textbf{Sijie Min} \\
\textbf{ Software Engineering Processes}  \hfill \textbf{40152234} \\
\hfill Repository address : https://github.com/Dakatsu/SOEN6011Calculator
\\\\\\\\\\
\section*{Unit Test Case for F5 Function}
The unit test cases for $ab^x$ function is done using related functions of \textbf{JUnit 4} \\\\\\
 \begin{center}
\textbf{Test Case : F5\_UnitTestCase\_1}\\\\
\begin{tabular}{ll}
\textbf{Test Case ID} &\ F5testF5\\\
\textbf{Requirement ID} & F5-R1 \\
\textbf{Action} & 
\begin{tabular}[c]{@{}l@{}}Test $ab^x$. a is set to 0, check the output result; x input is 0, check the output result
\end{tabular} \\
\textbf{Input(s) } & a=0,b=19,x=2; a=2,b=10,x=0 \\
\textbf{Expected Output } & 0; 2 (there are 2 sets of inputs so 2 sets of outputs,0 and 2)\\
\textbf{Actual Output } & 0; 2(there are 2 sets of inputs so 2 sets of outputs,0 and 2) \\
\textbf{Test Result } & Success \\
\end{tabular}
  \end{center}
 
 \begin{center}
\textbf{Test Case : F5\_UnitTestCase\_2}\\\\
\begin{tabular}{ll}
\textbf{Test Case ID} &\ F5testF5PositiveX\\\
\textbf{Requirement ID} & F5-R2 \\
\textbf{Action} & 
\begin{tabular}[c]{@{}l@{}}Test $ab^x$.the input of x is a positive number, check the output result
\end{tabular} \\
\textbf{Input(s) } & a=1.0,b=3.4,x=5.6 \\
\textbf{Expected Output } & 946.8516393\\
\textbf{Actual Output } & 946.8516393 \\
\textbf{Test Result } & Success \\
\end{tabular} 
  \end{center}
 
\begin{center}
\textbf{Test Case : F5\_UnitTestCase\_3}\\\\
\begin{tabular}{ll}
\textbf{Test Case ID} &\ F5testF5NegativeX\\\
\textbf{Requirement ID} & F5-R3 \\
\textbf{Action} & 
\begin{tabular}[c]{@{}l@{}}Test $ab^x$.the input of x is a negative number, check the output result
\end{tabular} \\
\textbf{Input(s) } & a=1.0,b=3.4,x=-5.6 \\
\textbf{Expected Output } & 0.0021122\\
\textbf{Actual Output } & 0.0021122 \\
\textbf{Test Result } & Success \\
\end{tabular} 
  \end{center} 
\pagebreak 

\textbf{Test Case : F5\_UnitTestCase\_4}\\\\
\begin{tabular}{ll}
\textbf{Test Case ID} &\ F5testF5NegativeX\\\
\textbf{Requirement ID} & F5-R4 \\
\textbf{Action} & 
\begin{tabular}[c]{@{}l@{}}Test power(double,int) function
\end{tabular} \\
\textbf{Input(s) } & power(1.6,7) \\
\textbf{Expected Output } & 26.8435456\\
\textbf{Actual Output } & 26.8435456 \\
\textbf{Test Result } & Success \\
\end{tabular} 
 
  

\textbf{Test Case : F5\_UnitTestCase\_4}\\\\
\begin{tabular}{ll}
\textbf{Test Case ID} &\ F5testF5NegativeX\\\
\textbf{Requirement ID} & F5-R5 \\
\textbf{Action} & 
\begin{tabular}[c]{@{}l@{}}Test power(double,int) function
\end{tabular} \\
\textbf{Input(s) } & power(1.6,7) \\
\textbf{Expected Output } & 26.8435456\\
\textbf{Actual Output } & 26.8435456 \\
\textbf{Test Result } & Success \\
\end{tabular} 


\textbf{Test Case : F5\_UnitTestCase\_5}\\\\
\begin{tabular}{ll}
\textbf{Test Case ID} &\ F5testDecimalPower\\
\textbf{Requirement ID} & F5-R6 \\
\textbf{Action} & 
\begin{tabular}[c]{@{}l@{}}Test power(double,double) function
\end{tabular} \\
\textbf{Input(s) } & power(5.6, 7.5) \\
\textbf{Expected Output } & 408705.2369134\\
\textbf{Actual Output } & 408705.2369134 \\
\textbf{Test Result } & Success \\
\end{tabular} 
\begin{figure}[htp]
    \centering
    \includegraphics[width=16cm]{F5p6}
    \caption{Figure: Unit Testing results for function $ab^x$}
    \label{fig:galaxy}
\end{figure}


  
  
\pagebreak

\section*{\centering{PROBLEM 6 - F7 : \(x^y\)}}
\normalsize {SOEN 6011 - Summer 2021} \hfill \textbf{Manimaran Palani} \\
\textbf{ Software Engineering Processes}  \hfill \textbf{40167543} \\
\hfill Repository address : https://github.com/Dakatsu/SOEN6011Calculator
\\
\section*{\textbf{Problem 6 - Unit Test Case Description}}
This section presents the unit test cases implemented using \textbf{JUnit4} for Super Calculator \\(F7-Power Function) which are traceable to requirements.\\\\\\
\textbf{Test Case : F7\_TestCase\_1}\\\\
\begin{tabular}{ll}
\textbf{Test Case ID} & F7\_TestCase\_1 \\
\textbf{Requirement ID} & F7-R1 \\
\textbf{Action} & 
\begin{tabular}[c]{@{}l@{}}The user inputs a base input and click power function button followed \\by giving exponent input and click result(=) button. \\
\end{tabular} \\
\textbf{Input(s) } & base = 0.0, exponent = 0.0 \\
\textbf{Expected Output } & 1.0 \\
\textbf{Actual Output } & 1.0 \\
\textbf{Test Result } & Success \\
\end{tabular}
\\\\\\\\
\textbf{Test Case : F7\_TestCase\_2}\\\\
\begin{tabular}{ll}
\textbf{Test Case ID} & F7\_TestCase\_2 \\
\textbf{Requirement ID} & F7-R2 \\
\textbf{Action} & 
\begin{tabular}[c]{@{}l@{}}The user inputs a base input and click power function button followed \\by giving exponent input and click result(=) button. \\
\end{tabular} \\
\textbf{Input(s) } & base = 0.0, exponent = 3.0 \\
\textbf{Expected Output } & 0.0 \\
\textbf{Actual Output } & 0.0 \\
\textbf{Test Result } & Success \\
\end{tabular}
\\\\\\\\\\\\\\\\\\\\\\\\\
\textbf{Test Case : F7\_TestCase\_3}\\\\
\begin{tabular}{ll}
\textbf{Test Case ID} & F7\_TestCase\_3 \\
\textbf{Requirement ID} & F7-R3 \\
\textbf{Action} & 
\begin{tabular}[c]{@{}l@{}}The user inputs a base input and click power function button followed \\by giving exponent input and click result(=) button. \\
\end{tabular} \\
\textbf{Input(s) } & base = 7.0, exponent = 0.0 \\
\textbf{Expected Output } & 1.0 \\
\textbf{Actual Output } & 1.0 \\
\textbf{Test Result } & Success \\
\end{tabular}
\\\\\\\\
\textbf{Test Case : F7\_TestCase\_4}\\\\
\begin{tabular}{ll}
\textbf{Test Case ID} & F7\_TestCase\_4 \\
\textbf{Requirement ID} & F7-R4 \\
\textbf{Action} & 
\begin{tabular}[c]{@{}l@{}}The user inputs a base input and click power function button followed \\by giving exponent input and click result(=) button. \\
\end{tabular} \\
\textbf{Input(s) } & base = -4.0, exponent = 0.0 \\
\textbf{Expected Output } & 1.0 \\
\textbf{Actual Output } & 1.0 \\
\textbf{Test Result } & Success \\
\end{tabular}
\\\\\\\\\\\\
\textbf{Test Case : F7\_TestCase\_5}\\\\
\begin{tabular}{ll}
\textbf{Test Case ID} & F7\_TestCase\_5 \\
\textbf{Requirement ID} & F7-R5 \\
\textbf{Action} & 
\begin{tabular}[c]{@{}l@{}}The user inputs a base input and click power function button followed \\by giving exponent input and click result(=) button. \\
\end{tabular} \\
\textbf{Input(s) } & base = 7.0, exponent = 1.0 \\
\textbf{Expected Output } & 7.0 \\
\textbf{Actual Output } & 7.0 \\
\textbf{Test Result } & Success \\
\end{tabular}
\\\\\\\\\\\\\\\\\\\\\\
\textbf{Test Case : F7\_TestCase\_6}\\\\
\begin{tabular}{ll}
\textbf{Test Case ID} & F7\_TestCase\_6 \\
\textbf{Requirement ID} & F7-R6 \\
\textbf{Action} & 
\begin{tabular}[c]{@{}l@{}}The user inputs a base input and click power function button followed \\by giving exponent input and click result(=) button. \\
\end{tabular} \\
\textbf{Input(s) } & base = 5, exponent = 9 \\
\textbf{Expected Output } & 1953125.0 \\
\textbf{Actual Output } & 1953125.0 \\
\textbf{Test Result } & Success \\
\end{tabular}
\\\\\\\\
\textbf{Test Case : F7\_TestCase\_7}\\\\
\begin{tabular}{ll}
\textbf{Test Case ID} & F7\_TestCase\_7 \\
\textbf{Requirement ID} & F7-R6 \\
\textbf{Action} & 
\begin{tabular}[c]{@{}l@{}}The user inputs a base input and click power function button followed \\by giving exponent input and click result(=) button. \\
\end{tabular} \\
\textbf{Input(s) } & base = -3, exponent = 4.4     \\
\textbf{Expected Output } & 3.1631 \\
\textbf{Actual Output } & 3.1631 \\
\textbf{Test Result } & Success \\
\end{tabular}
\\\\\\
\textbf{Test Case : F7\_TestCase\_8}\\\\
\begin{tabular}{ll}
\textbf{Test Case ID} & F7\_TestCase\_8 \\
\textbf{Requirement ID} & F7-R6 \\
\textbf{Action} & 
\begin{tabular}[c]{@{}l@{}}The user inputs a base input and click power function button followed \\by giving exponent input and click result(=) button. \\
\end{tabular} \\
\textbf{Input(s) } & base = -9, exponent = 3 \\
\textbf{Expected Output } & -729 \\
\textbf{Actual Output } & -729 \\
\textbf{Test Result } & Success \\
\end{tabular}\\\\\\\\\\\\\\\\\\\\\\\\\\\\
\textbf{Test Case Results for F7}
\begin{figure}[htb]
\begin{center}
\includegraphics[width=13cm]{TestCases_Results_F7}
  \centering
  \caption{ Test case result of function F7 : \(x^y\)  using Junit4
}
\end{center}
\end{figure}
\end{document}